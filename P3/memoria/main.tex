%%%%%%%%%%%%%%%%%%%%%%%%%%%%%%%%%%%%%%%%%
% Short Sectioned Assignment LaTeX Template Version 1.0 (5/5/12)
% This template has been downloaded from: http://www.LaTeXTemplates.com
% Original author:  Frits Wenneker (http://www.howtotex.com)
% License: CC BY-NC-SA 3.0 (http://creativecommons.org/licenses/by-nc-sa/3.0/)
%%%%%%%%%%%%%%%%%%%%%%%%%%%%%%%%%%%%%%%%%

%----------------------------------------------------------------------------------------
%	PACKAGES AND OTHER DOCUMENT CONFIGURATIONS
%----------------------------------------------------------------------------------------

\documentclass[paper=a4, fontsize=11pt]{scrartcl} % A4 paper and 11pt font size

% ---- Entrada y salida de texto -----

\usepackage[T1]{fontenc} % Use 8-bit encoding that has 256 glyphs
\usepackage[utf8]{inputenc}

% ---- Idioma --------

\usepackage[spanish, es-tabla]{babel} % Selecciona el español para palabras introducidas automáticamente, p.ej. "septiembre" en la fecha y especifica que se use la palabra Tabla en vez de Cuadro

% ---- Otros paquetes ----

\usepackage{amsmath,amsfonts,amsthm} % Math packages
\usepackage{graphics,graphicx, floatrow} %para incluir imágenes y notas en las imágenes
\usepackage{graphics,graphicx, float} %para incluir imágenes y colocarlas
\usepackage{hyperref} % url in references

% Para hacer tablas comlejas
\usepackage{multirow}
\usepackage{threeparttable}

\usepackage{fancyhdr} % Custom headers and footers
\pagestyle{fancyplain} % Makes all pages in the document conform to the custom headers and footers
\fancyhead{} % No page header - if you want one, create it in the same way as the footers below
\fancyfoot[L]{} % Empty left footer
\fancyfoot[C]{} % Empty center footer
\fancyfoot[R]{\thepage} % Page numbering for right footer
\renewcommand{\headrulewidth}{0pt} % Remove header underlines
\renewcommand{\footrulewidth}{0pt} % Remove footer underlines
\setlength{\headheight}{13.6pt} % Customize the height of the header

\numberwithin{equation}{section} % Number equations within sections (i.e. 1.1, 1.2, 2.1, 2.2 instead of 1, 2, 3, 4)
\numberwithin{figure}{section} % Number figures within sections (i.e. 1.1, 1.2, 2.1, 2.2 instead of 1, 2, 3, 4)
\numberwithin{table}{section} % Number tables within sections (i.e. 1.1, 1.2, 2.1, 2.2 instead of 1, 2, 3, 4)

\setlength\parindent{0pt} % Removes all indentation from paragraphs - comment this line for an assignment with lots of text

\newcommand{\horrule}[1]{\rule{\linewidth}{#1}} % Create horizontal rule command with 1 argument of height

\usepackage{textcomp}
\usepackage{hyperref}

%----------------------------------------------------------------------------------------
%	DATOS
%----------------------------------------------------------------------------------------

\newcommand{\myName}{Francisco Javier Bolívar Lupiáñez}
\newcommand{\myDegree}{Máster en Ingeniería Informática}
\newcommand{\myFaculty}{E. T. S. de Ingenierías Informática y de Telecomunicación}
\newcommand{\myDepartment}{Ciencias de la Computación e Inteligencia Artificial}
\newcommand{\myUniversity}{\protect{Universidad de Granada}}
\newcommand{\myLocation}{Granada}
\newcommand{\myTime}{\today}
\newcommand{\myTitle}{Práctica 3}
\newcommand{\mySubtitle}{Bases de datos NoSQL}
\newcommand{\mySubject}{Cloud Computing: Servicios y Aplicaciones}
\newcommand{\myYear}{2016-2017}

%----------------------------------------------------------------------------------------
%	PORTADA
%----------------------------------------------------------------------------------------


\title{	
	\normalfont \normalsize 
	\textsc{\textbf{\mySubject \space (\myYear)} \\ \myDepartment} \\[20pt] % Your university, school and/or department name(s)
	\textsc{\myDegree \\[10pt] \myFaculty \\ \myUniversity} \\[25pt]
	\horrule{0.5pt} \\[0.4cm] % Thin top horizontal rule
	\huge \myTitle: \mySubtitle \\ % The assignment title
	\horrule{2pt} \\[0.5cm] % Thick bottom horizontal rule
	\normalfont \normalsize
}

\author{\myName} % Nombre y apellidos

\date{\myTime} % Incluye la fecha actual
%----------------------------------------------------------------------------------------
%	INDICE
%----------------------------------------------------------------------------------------

\begin{document}
	
\lstset {
	basicstyle=\scriptsize,
	frame=single,
	backgroundcolor=\color{grey},
	breaklines=true
}
	
\setcounter{page}{0}

\maketitle % Muestra el Título
\thispagestyle{empty}

\newpage %inserta un salto de página

\tableofcontents % para generar el índice de contenidos

%\listoffigures

\newpage

%----------------------------------------------------------------------------------------
%	DOCUMENTO
%----------------------------------------------------------------------------------------

\section{Introducción}

El objetivo de esta práctica es familiarizarse con el uso de un sistema de gestión de bases de datos en entornos Big Data. Para ello, haremos uso de la aplicación más conocida como es MongoDB.
\\ \\
Para constatar el manejo de la herramienta anterior, el alumno deberá las realizar tareas que se describen a continuación y entregawr documentación escribiendo las tareas realizadas.

\section{Objetivo nº 1. Consulta de Documentos}

Crear la colección \texttt{pedidos} en cada BD asociada a vuestro usuario, sobre la que se realizarán diversas operaciones CRUD. Para crear la colección abre y ejecuta el script \texttt{insertar\_pedidos.js} (accesible en \texttt{/tmp/mongo}). Las tareas a realizar son las siguientes:

\subsection{Ejercicio 1}

Visualiza la colección \texttt{pedidos} y familiarízate con ella. Observa los distintos tipos de datos y sus estructuras dispares

\begin{lstlisting}
db.pedidos.find().pretty()
\end{lstlisting}

\begin{lstlisting}
{
  "_id" : ObjectId("58f64cf627193f23545c3e70"),
  "id_cliente" : 1111,
  "Nombre" : "Pedro Ramirez",
  "Direccion" : "Calle Los Romeros 14",
  "Localidad" : "Sevilla",
  "Fnacimiento" : ISODate("1963-04-03T00:00:00Z"),
  "Facturacion" : 5000,
  "Pedidos" : [
    {
      "id_pedido" : 1,
      "Productos" : [
        {
          "id_producto" : 1,
          "Nombre" : "Pentium IV",
          "Fabricante" : "Intel",
          "Precio_unidad" : 390,
          "Cantidad" : 1
        },
        {
          "id_producto" : 2,
          "Nombre" : "Tablet 8 pulgadas",
          "Precio_unidad" : 95,
          "Cantidad" : 1
        }
      ]
    },
    {
      "id_pedido" : 2,
      "Productos" : [
        {
          "id_producto" : 77,
          "Nombre" : "Impresora Laser",
          "Fabricante" : "Canon",
          "Precio_unidad" : 115,
          "Cantidad" : 3
        }
      ]
    }
  ]
}
{
  "_id" : ObjectId("58f64cf727193f23545c3e71"),
  "id_cliente" : 2222,
  "Nombre" : "Juan Gomez",
  "Direccion" : "Perpetuo Socorro 9",
  "Localidad" : "Salamanca",
  "Fnacimiento" : ISODate("1960-08-17T00:00:00Z"),
  "Facturacion" : 6500,
  "Pedidos" : [
    {
      "id_pedido" : 1,
      "Productos" : [
        {
          "id_producto" : 1,
          "Nombre" : "Pentium IV",
          "Fabricante" : "Intel",
          "Precio_unidad" : 100,
          "Cantidad" : 1
        },
        {
          "id_producto" : 42,
          "Nombre" : "Portatil ASM Mod. 254",
          "Fabricante" : "Intel",
          "Precio_unidad" : 455,
          "Cantidad" : 2
        },
        {
          "id_producto" : 27,
          "Nombre" : "Cable USB",
          "Precio_unidad" : 11,
          "Cantidad" : 12
        }
      ]
    },
    {
      "id_pedido" : 2,
      "Productos" : [
        {
          "id_producto" : 77,
          "Nombre" : "Impresora Laser",
          "Fabricante" : "Canon",
          "Precio_unidad" : 128,
          "Cantidad" : 3
        },
        {
          "id_producto" : 42,
          "Nombre" : "Portatil ASM Mod. 254",
          "Fabricante" : "Intel",
          "Precio_unidad" : 451,
          "Cantidad" : 5
        },
        {
          "id_producto" : 21,
          "Nombre" : "Disco Duro 500GB",
          "Precio_unidad" : 99,
          "Cantidad" : 10
        }
      ]
    },
    {
      "id_pedido" : 3,
      "Productos" : [
        {
          "id_producto" : 1,
          "Nombre" : "Pentium IV",
          "Fabricante" : "Intel",
          "Precio_unidad" : 94,
          "Cantidad" : 5
        },
        {
          "id_producto" : 95,
          "Nombre" : "SAI 5H Mod. 258",
          "Precio_unidad" : 213,
          "Cantidad" : 2
        },
        {
          "id_producto" : 21,
          "Precio_unidad" : 66,
          "Nombre" : "Disco Duro 500GB",
          "Cantidad" : 10
        }
      ]
    }
  ]
}
{
  "_id" : ObjectId("58f64cf727193f23545c3e72"),
  "id_cliente" : 3333,
  "Nombre" : "Carlos Montes",
  "Direccion" : "Salsipuedes 13",
  "Localidad" : "Jaen",
  "Fnacimiento" : ISODate("1967-11-25T00:00:00Z"),
  "Facturacion" : 8000
}
{
  "_id" : ObjectId("58f64cf727193f23545c3e73"),
  "id_cliente" : 4444,
  "Nombre" : "Carmelo Coton",
  "Direccion" : "La Luna 103",
  "Localidad" : "Jaen",
  "Fnacimiento" : ISODate("1969-01-06T00:00:00Z"),
  "Facturacion" : 12300
}
{
  "_id" : ObjectId("58f64cf727193f23545c3e74"),
  "id_cliente" : 5555,
  "Nombre" : "Cristina Miralles",
  "Direccion" : "San Fernando 28",
  "Localidad" : "Granada",
  "Fnacimiento" : ISODate("1970-07-12T00:00:00Z"),
  "Facturacion" : 16500,
  "Pedidos" : [
    {
      "id_pedido" : 1,
      "Productos" : [
        {
          "id_producto" : 95,
          "Nombre" : "SAI 5H Mod. 258",
          "Precio_unidad" : 211,
          "Cantidad" : 2
        },
        {
          "id_producto" : 42,
          "Nombre" : "Portatil ASM Mod. 254",
          "Precio_unidad" : 460,
          "Fabricante" : "Intel",
          "Cantidad" : 2
        },
        {
          "id_producto" : 77,
          "Nombre" : "Impresora Laser",
          "Fabricante" : "Canon",
          "Precio_unidad" : 119,
          "Cantidad" : 2
        }
      ]
    }
  ]
}
{
  "_id" : ObjectId("58f64cf727193f23545c3e75"),
  "id_cliente" : 6666,
  "Nombre" : "Chema Pamundi",
  "Direccion" : "Recogidas 54",
  "Localidad" : "Granada",
  "Fnacimiento" : ISODate("1969-02-04T00:00:00Z"),
  "Facturacion" : 5000
}
{
  "_id" : ObjectId("58f64cf727193f23545c3e76"),
  "id_cliente" : 777,
  "Nombre" : "Alberto Matero",
  "Direccion" : "Pelayo 4",
  "Localidad" : "Sevilla",
  "Facturacion" : 2500,
  "Pedidos" : null
}
\end{lstlisting}

\subsection{Ejercicio 2}

Visualiza sólo el primer documento de la colección. Utiliza los métodos \texttt{.limit()} y \texttt{.findOne()}

\begin{lstlisting}
db.pedidos.findOne();
// o
db.pedidos.find().limit(1).pretty();
\end{lstlisting}

\begin{lstlisting}
{
  "_id" : ObjectId("58f64cf627193f23545c3e70"),
  "id_cliente" : 1111,
  "Nombre" : "Pedro Ramirez",
  "Direccion" : "Calle Los Romeros 14",
  "Localidad" : "Sevilla",
  "Fnacimiento" : ISODate("1963-04-03T00:00:00Z"),
  "Facturacion" : 5000,
  "Pedidos" : [
    {
      "id_pedido" : 1,
      "Productos" : [
        {
          "id_producto" : 1,
          "Nombre" : "Pentium IV",
          "Fabricante" : "Intel",
          "Precio_unidad" : 390,
          "Cantidad" : 1
        },
        {
          "id_producto" : 2,
          "Nombre" : "Tablet 8 pulgadas",
          "Precio_unidad" : 95,
          "Cantidad" : 1
        }
      ]
    },
    {
      "id_pedido" : 2,
      "Productos" : [
        {
          "id_producto" : 77,
          "Nombre" : "Impresora Laser",
          "Fabricante" : "Canon",
          "Precio_unidad" : 115,
          "Cantidad" : 3
        }
      ]
    }
  ]
}	
\end{lstlisting}

\subsection{Ejercicio 3}

Visualiza el cliente con \texttt{id\_cliente = 2222}

\begin{lstlisting}
db.pedidos.find(
  {"id_cliente": 2222}
).pretty();
\end{lstlisting}

\begin{lstlisting}
{
  "_id" : ObjectId("58f64cf727193f23545c3e71"),
  "id_cliente" : 2222,
  "Nombre" : "Juan Gomez",
  "Direccion" : "Perpetuo Socorro 9",
  "Localidad" : "Salamanca",
  "Fnacimiento" : ISODate("1960-08-17T00:00:00Z"),
  "Facturacion" : 6500,
  "Pedidos" : [
    {
      "id_pedido" : 1,
      "Productos" : [
        {
          "id_producto" : 1,
          "Nombre" : "Pentium IV",
          "Fabricante" : "Intel",
          "Precio_unidad" : 100,
          "Cantidad" : 1
        },
        {
          "id_producto" : 42,
          "Nombre" : "Portatil ASM Mod. 254",
          "Fabricante" : "Intel",
          "Precio_unidad" : 455,
          "Cantidad" : 2
        },
        {
          "id_producto" : 27,
          "Nombre" : "Cable USB",
          "Precio_unidad" : 11,
          "Cantidad" : 12
        }
      ]
    },
    {
      "id_pedido" : 2,
      "Productos" : [
        {
          "id_producto" : 77,
          "Nombre" : "Impresora Laser",
          "Fabricante" : "Canon",
          "Precio_unidad" : 128,
          "Cantidad" : 3
        },
        {
          "id_producto" : 42,
          "Nombre" : "Portatil ASM Mod. 254",
          "Fabricante" : "Intel",
          "Precio_unidad" : 451,
          "Cantidad" : 5
        },
        {
          "id_producto" : 21,
          "Nombre" : "Disco Duro 500GB",
          "Precio_unidad" : 99,
          "Cantidad" : 10
        }
      ]
    },
    {
      "id_pedido" : 3,
      "Productos" : [
        {
          "id_producto" : 1,
          "Nombre" : "Pentium IV",
          "Fabricante" : "Intel",
          "Precio_unidad" : 94,
          "Cantidad" : 5
        },
        {
          "id_producto" : 95,
          "Nombre" : "SAI 5H Mod. 258",
          "Precio_unidad" : 213,
          "Cantidad" : 2
        },
        {
          "id_producto" : 21,
          "Precio_unidad" : 66,
          "Nombre" : "Disco Duro 500GB",
          "Cantidad" : 10
        }
      ]
    }
  ]
}
	
\end{lstlisting}

\subsection{Ejercicio 4}

Visualiza los clientes que hayan pedido algún producto de más de 94 euros

\begin{lstlisting}
db.pedidos.find(
  {"Pedidos.Productos.Precio_unidad": {$gt: 94}}
).pretty();
\end{lstlisting}

\begin{lstlisting}
{
  "_id" : ObjectId("58f64cf627193f23545c3e70"),
  "id_cliente" : 1111,
  "Nombre" : "Pedro Ramirez",
  "Direccion" : "Calle Los Romeros 14",
  "Localidad" : "Sevilla",
  "Fnacimiento" : ISODate("1963-04-03T00:00:00Z"),
  "Facturacion" : 5000,
  "Pedidos" : [
    {
      "id_pedido" : 1,
      "Productos" : [
        {
          "id_producto" : 1,
          "Nombre" : "Pentium IV",
          "Fabricante" : "Intel",
          "Precio_unidad" : 390,
          "Cantidad" : 1
        },
        {
          "id_producto" : 2,
          "Nombre" : "Tablet 8 pulgadas",
          "Precio_unidad" : 95,
          "Cantidad" : 1
        }
      ]
    },
    {
      "id_pedido" : 2,
      "Productos" : [
        {
          "id_producto" : 77,
          "Nombre" : "Impresora Laser",
          "Fabricante" : "Canon",
          "Precio_unidad" : 115,
          "Cantidad" : 3
        }
      ]
    }
  ]
}
{
  "_id" : ObjectId("58f64cf727193f23545c3e71"),
  "id_cliente" : 2222,
  "Nombre" : "Juan Gomez",
  "Direccion" : "Perpetuo Socorro 9",
  "Localidad" : "Salamanca",
  "Fnacimiento" : ISODate("1960-08-17T00:00:00Z"),
  "Facturacion" : 6500,
  "Pedidos" : [
    {
      "id_pedido" : 1,
      "Productos" : [
        {
          "id_producto" : 1,
          "Nombre" : "Pentium IV",
          "Fabricante" : "Intel",
          "Precio_unidad" : 100,
          "Cantidad" : 1
        },
        {
          "id_producto" : 42,
          "Nombre" : "Portatil ASM Mod. 254",
          "Fabricante" : "Intel",
          "Precio_unidad" : 455,
          "Cantidad" : 2
        },
        {
          "id_producto" : 27,
          "Nombre" : "Cable USB",
          "Precio_unidad" : 11,
          "Cantidad" : 12
        }
      ]
    },
    {
      "id_pedido" : 2,
      "Productos" : [
        {
          "id_producto" : 77,
          "Nombre" : "Impresora Laser",
          "Fabricante" : "Canon",
          "Precio_unidad" : 128,
          "Cantidad" : 3
        },
        {
          "id_producto" : 42,
          "Nombre" : "Portatil ASM Mod. 254",
          "Fabricante" : "Intel",
          "Precio_unidad" : 451,
          "Cantidad" : 5
        },
        {
          "id_producto" : 21,
          "Nombre" : "Disco Duro 500GB",
          "Precio_unidad" : 99,
          "Cantidad" : 10
        }
      ]
    },
    {
      "id_pedido" : 3,
      "Productos" : [
        {
          "id_producto" : 1,
          "Nombre" : "Pentium IV",
          "Fabricante" : "Intel",
          "Precio_unidad" : 94,
          "Cantidad" : 5
        },
        {
          "id_producto" : 95,
          "Nombre" : "SAI 5H Mod. 258",
          "Precio_unidad" : 213,
          "Cantidad" : 2
        },
        {
          "id_producto" : 21,
          "Precio_unidad" : 66,
          "Nombre" : "Disco Duro 500GB",
          "Cantidad" : 10
        }
      ]
    }
  ]
}
{
  "_id" : ObjectId("58f64cf727193f23545c3e74"),
  "id_cliente" : 5555,
  "Nombre" : "Cristina Miralles",
  "Direccion" : "San Fernando 28",
  "Localidad" : "Granada",
  "Fnacimiento" : ISODate("1970-07-12T00:00:00Z"),
  "Facturacion" : 16500,
  "Pedidos" : [
    {
      "id_pedido" : 1,
      "Productos" : [
        {
          "id_producto" : 95,
          "Nombre" : "SAI 5H Mod. 258",
          "Precio_unidad" : 211,
          "Cantidad" : 2
        },
        {
          "id_producto" : 42,
          "Nombre" : "Portatil ASM Mod. 254",
          "Precio_unidad" : 460,
          "Fabricante" : "Intel",
          "Cantidad" : 2
        },
        {
          "id_producto" : 77,
          "Nombre" : "Impresora Laser",
          "Fabricante" : "Canon",
          "Precio_unidad" : 119,
          "Cantidad" : 2
        }
      ]
    }
  ]
}
\end{lstlisting}

\subsection{Ejercicio 5}

Visualiza los clientes de Jaén o Salamanca (excluye los datos de los pedidos). Utiliza los operador \texttt{\$or} e \texttt{\$in}

\begin{lstlisting}
db.pedidos.find(
  {$or:[{"Localidad": "Salamanca"}, {"Localidad": "Jaen"}]}, 
  {"Pedidos": 0}
).pretty();
// o
db.pedidos.find(
  {"Localidad": {$in: ["Salamanca", "Jaen"]}}, 
  {"Pedidos": 0}
).pretty();
\end{lstlisting}

\begin{lstlisting}
{
  "_id" : ObjectId("58f64cf727193f23545c3e71"),
  "id_cliente" : 2222,
  "Nombre" : "Juan Gomez",
  "Direccion" : "Perpetuo Socorro 9",
  "Localidad" : "Salamanca",
  "Fnacimiento" : ISODate("1960-08-17T00:00:00Z"),
  "Facturacion" : 6500
}
{
  "_id" : ObjectId("58f64cf727193f23545c3e72"),
  "id_cliente" : 3333,
  "Nombre" : "Carlos Montes",
  "Direccion" : "Salsipuedes 13",
  "Localidad" : "Jaen",
  "Fnacimiento" : ISODate("1967-11-25T00:00:00Z"),
  "Facturacion" : 8000
}
{
  "_id" : ObjectId("58f64cf727193f23545c3e73"),
  "id_cliente" : 4444,
  "Nombre" : "Carmelo Coton",
  "Direccion" : "La Luna 103",
  "Localidad" : "Jaen",
  "Fnacimiento" : ISODate("1969-01-06T00:00:00Z"),
  "Facturacion" : 12300
}
\end{lstlisting}

\subsection{Ejercicio 6}

Visualiza los clientes no tienen campo \texttt{pedidos}

\begin{lstlisting}
db.pedidos.find(
  {"Pedidos": {$exists: false}}
).pretty();
\end{lstlisting}

\begin{lstlisting}
{
  "_id" : ObjectId("58f64cf727193f23545c3e72"),
  "id_cliente" : 3333,
  "Nombre" : "Carlos Montes",
  "Direccion" : "Salsipuedes 13",
  "Localidad" : "Jaen",
  "Fnacimiento" : ISODate("1967-11-25T00:00:00Z"),
  "Facturacion" : 8000
}
{
  "_id" : ObjectId("58f64cf727193f23545c3e73"),
  "id_cliente" : 4444,
  "Nombre" : "Carmelo Coton",
  "Direccion" : "La Luna 103",
  "Localidad" : "Jaen",
  "Fnacimiento" : ISODate("1969-01-06T00:00:00Z"),
  "Facturacion" : 12300
}
{
  "_id" : ObjectId("58f64cf727193f23545c3e75"),
  "id_cliente" : 6666,
  "Nombre" : "Chema Pamundi",
  "Direccion" : "Recogidas 54",
  "Localidad" : "Granada",
  "Fnacimiento" : ISODate("1969-02-04T00:00:00Z"),
  "Facturacion" : 5000
}
\end{lstlisting}

\subsection{Ejercicio 7}

Visualiza los clientes que hayan nacido en 1963

\begin{lstlisting}
db.pedidos.find(
  {"Fnacimiento": {$gte: new Date(1963, 1, 1), $lt: new Date(1964, 1, 1)}}
).pretty();
\end{lstlisting}

\begin{lstlisting}
{
  "_id" : ObjectId("58f64cf627193f23545c3e70"),
  "id_cliente" : 1111,
  "Nombre" : "Pedro Ramirez",
  "Direccion" : "Calle Los Romeros 14",
  "Localidad" : "Sevilla",
  "Fnacimiento" : ISODate("1963-04-03T00:00:00Z"),
  "Facturacion" : 5000,
  "Pedidos" : [
    {
      "id_pedido" : 1,
      "Productos" : [
        {
          "id_producto" : 1,
          "Nombre" : "Pentium IV",
          "Fabricante" : "Intel",
          "Precio_unidad" : 390,
          "Cantidad" : 1
        },
        {
          "id_producto" : 2,
          "Nombre" : "Tablet 8 pulgadas",
          "Precio_unidad" : 95,
          "Cantidad" : 1
        }
      ]
    },
    {
      "id_pedido" : 2,
      "Productos" : [
        {
          "id_producto" : 77,
          "Nombre" : "Impresora Laser",
          "Fabricante" : "Canon",
          "Precio_unidad" : 115,
          "Cantidad" : 3
        }
      ]
    }
  ]
}
\end{lstlisting}

\subsection{Ejercicio 8}

Visualiza los clientes que hayan pedido algún producto fabricado por Canon y algún producto cuyo precio sea inferior a 15 euros

\begin{lstlisting}
db.pedidos.find(
  {"Pedidos.Productos.Fabricante": "Canon", 
  "Pedidos.Productos.Precio_unidad": {$lt: 15}}
).pretty();
\end{lstlisting}

\begin{lstlisting}
{
  "_id" : ObjectId("58f64cf727193f23545c3e71"),
  "id_cliente" : 2222,
  "Nombre" : "Juan Gomez",
  "Direccion" : "Perpetuo Socorro 9",
  "Localidad" : "Salamanca",
  "Fnacimiento" : ISODate("1960-08-17T00:00:00Z"),
  "Facturacion" : 6500,
  "Pedidos" : [
    {
      "id_pedido" : 1,
      "Productos" : [
        {
          "id_producto" : 1,
          "Nombre" : "Pentium IV",
          "Fabricante" : "Intel",
          "Precio_unidad" : 100,
          "Cantidad" : 1
        },
        {
          "id_producto" : 42,
          "Nombre" : "Portatil ASM Mod. 254",
          "Fabricante" : "Intel",
          "Precio_unidad" : 455,
          "Cantidad" : 2
        },
        {
          "id_producto" : 27,
          "Nombre" : "Cable USB",
          "Precio_unidad" : 11,
          "Cantidad" : 12
        }
      ]
    },
    {
      "id_pedido" : 2,
      "Productos" : [
        {
          "id_producto" : 77,
          "Nombre" : "Impresora Laser",
          "Fabricante" : "Canon",
          "Precio_unidad" : 128,
          "Cantidad" : 3
        },
        {
          "id_producto" : 42,
          "Nombre" : "Portatil ASM Mod. 254",
          "Fabricante" : "Intel",
          "Precio_unidad" : 451,
          "Cantidad" : 5
        },
        {
          "id_producto" : 21,
          "Nombre" : "Disco Duro 500GB",
          "Precio_unidad" : 99,
          "Cantidad" : 10
        }
      ]
    },
    {
      "id_pedido" : 3,
      "Productos" : [
        {
          "id_producto" : 1,
          "Nombre" : "Pentium IV",
          "Fabricante" : "Intel",
          "Precio_unidad" : 94,
          "Cantidad" : 5
        },
        {
          "id_producto" : 95,
          "Nombre" : "SAI 5H Mod. 258",
          "Precio_unidad" : 213,
          "Cantidad" : 2
        },
        {
          "id_producto" : 21,
          "Precio_unidad" : 66,
          "Nombre" : "Disco Duro 500GB",
          "Cantidad" : 10
        }
      ]
    }
  ]
}
\end{lstlisting}

\subsection{Ejercicio 9}

Datos personales (id\_cliente, Nombre, Direccion, Localidad y Fnacimiento) de los clientes cuyo nombre empieza por la cadena "c" (No distinguir entre mayusculas y minúsculas)

\begin{lstlisting}
db.pedidos.find(
  {"Nombre": /^c/i}, 
  {"_id": 0, "id_cliente": 1, "Nombre": 1, "Direccion": 1, "Localidad": 1, 
  "Fnacimiento": 1,}
).pretty();
\end{lstlisting}

\begin{lstlisting}
{
  "id_cliente" : 3333,
  "Nombre" : "Carlos Montes",
  "Direccion" : "Salsipuedes 13",
  "Localidad" : "Jaen",
  "Fnacimiento" : ISODate("1967-11-25T00:00:00Z")
}
{
  "id_cliente" : 4444,
  "Nombre" : "Carmelo Coton",
  "Direccion" : "La Luna 103",
  "Localidad" : "Jaen",
  "Fnacimiento" : ISODate("1969-01-06T00:00:00Z")
}
{
  "id_cliente" : 5555,
  "Nombre" : "Cristina Miralles",
  "Direccion" : "San Fernando 28",
  "Localidad" : "Granada",
  "Fnacimiento" : ISODate("1970-07-12T00:00:00Z")
}
{
  "id_cliente" : 6666,
  "Nombre" : "Chema Pamundi",
  "Direccion" : "Recogidas 54",
  "Localidad" : "Granada",
  "Fnacimiento" : ISODate("1969-02-04T00:00:00Z")
}
\end{lstlisting}

\subsection{Ejercicio 10}

Visualiza los datos personales de los clientes (excluyendo \texttt{\_id}). Limita los documentos a 4

\begin{lstlisting}
db.pedidos.find(
  {}, 
  {"Pedidos": 0, "_id": 0}
).limit(4).pretty();
\end{lstlisting}

\begin{lstlisting}
{
  "id_cliente" : 1111,
  "Nombre" : "Pedro Ramirez",
  "Direccion" : "Calle Los Romeros 14",
  "Localidad" : "Sevilla",
  "Fnacimiento" : ISODate("1963-04-03T00:00:00Z"),
  "Facturacion" : 5000
}
{
  "id_cliente" : 2222,
  "Nombre" : "Juan Gomez",
  "Direccion" : "Perpetuo Socorro 9",
  "Localidad" : "Salamanca",
  "Fnacimiento" : ISODate("1960-08-17T00:00:00Z"),
  "Facturacion" : 6500
}
{
  "id_cliente" : 3333,
  "Nombre" : "Carlos Montes",
  "Direccion" : "Salsipuedes 13",
  "Localidad" : "Jaen",
  "Fnacimiento" : ISODate("1967-11-25T00:00:00Z"),
  "Facturacion" : 8000
}
{
  "id_cliente" : 4444,
  "Nombre" : "Carmelo Coton",
  "Direccion" : "La Luna 103",
  "Localidad" : "Jaen",
  "Fnacimiento" : ISODate("1969-01-06T00:00:00Z"),
  "Facturacion" : 12300
}
\end{lstlisting}

\subsection{Ejercicio 11}

Ídem anterior pero ordenando los documentos por Localidad (ascendente) e \texttt{id\_cliente} (descendente)

\begin{lstlisting}
db.pedidos.find(
  {}, 
  {"Pedidos": 0, "_id": 0}
).sort(
  {"Localidad": 1, "id_cliente": -1}
).limit(4).pretty();
\end{lstlisting}

\begin{lstlisting}
{
  "id_cliente" : 6666,
  "Nombre" : "Chema Pamundi",
  "Direccion" : "Recogidas 54",
  "Localidad" : "Granada",
  "Fnacimiento" : ISODate("1969-02-04T00:00:00Z"),
  "Facturacion" : 5000
}
{
  "id_cliente" : 5555,
  "Nombre" : "Cristina Miralles",
  "Direccion" : "San Fernando 28",
  "Localidad" : "Granada",
  "Fnacimiento" : ISODate("1970-07-12T00:00:00Z"),
  "Facturacion" : 16500
}
{
  "id_cliente" : 4444,
  "Nombre" : "Carmelo Coton",
  "Direccion" : "La Luna 103",
  "Localidad" : "Jaen",
  "Fnacimiento" : ISODate("1969-01-06T00:00:00Z"),
  "Facturacion" : 12300
}
{
  "id_cliente" : 3333,
  "Nombre" : "Carlos Montes",
  "Direccion" : "Salsipuedes 13",
  "Localidad" : "Jaen",
  "Fnacimiento" : ISODate("1967-11-25T00:00:00Z"),
  "Facturacion" : 8000
}

\end{lstlisting}

\section{Objetivo nº 2. Agregación}

A partir de la colección \texttt{pedidos} utilizaremos consultas más complejas por medio de los operadores de agregación (pipeline). Por facilidad se indica la consulta en formato SQL estándar. Las tareas a realizar en este caso obtener:

\subsection{Ejercicio 1}

Nº total de clientes

\begin{lstlisting}
db.pedidos.count();
\end{lstlisting}

\begin{lstlisting}
7
\end{lstlisting}

\subsection{Ejercicio 2}

Nº total de clientes de Jaén

\begin{lstlisting}
db.pedidos.find(
  {"Localidad": "Jaen"}
).count();
\end{lstlisting}

\begin{lstlisting}
2
\end{lstlisting}

\subsection{Ejercicio 3}

Facturación total clientes por localidad

\begin{lstlisting}
db.pedidos.aggregate(
  [
    {
      $group: 
      {
        "_id": "$Localidad", 
        facturacion_total: {$sum: "$Facturacion"}
      }
    }
  ]
);
\end{lstlisting}

\begin{lstlisting}
{ "_id" : "Granada", "facturacion_total" : 21500 }
{ "_id" : "Jaen", "facturacion_total" : 20300 }
{ "_id" : "Salamanca", "facturacion_total" : 6500 }
{ "_id" : "Sevilla", "facturacion_total" : 7500 }
\end{lstlisting}

\subsection{Ejercicio 4}

Facturación media de clientes por localidad para las localidades distintas a \texttt{"Jaen"} con facturación media mayor de 5000. Ordenación por Localidad descendente. Eliminar el \texttt{\_id} y poner el nombre en mayúsculas

\begin{lstlisting}
db.pedidos.aggregate(
  [
    {
      $group: 
      {
        "_id": "$Localidad", 
        facturacion_media: {$avg: "$Facturacion"}
      }
    }, 
    {
      $match: 
      {
        "_id": {$ne: "Jaen"}, 
        facturacion_media: {$gt: 5000}
      }
    }, 
    {
      $sort: 
      {
        "_id": -1
      }
    },
    {
      $project:
      {
        "_id": 0,
        "Localidad": {$toUpper: "$_id"},
        "facturacion_media": 1
      }
    }
  ]
);
\end{lstlisting}

\begin{lstlisting}
{ "facturacion_media" : 6500, "Localidad" : "SALAMANCA" }
{ "facturacion_media" : 10750, "Localidad" : "GRANADA" }
\end{lstlisting}

\subsection{Ejercicio 5}

Calcula la cantidad total facturada por cada cliente (uso de ``unwind'')

\begin{lstlisting}
db.pedidos.aggregate(
  [
    {
      $unwind: "$Pedidos"
    },
    {
      $unwind: "$Pedidos.Productos"
    },
    {
      $group: 
      {
        "_id": "$id_cliente", 
        facturacion_total: 
        {
          $sum: 
          {
            "$multiply": 
            [
              "$Pedidos.Productos.Precio_unidad",
              "$Pedidos.Productos.Cantidad"
            ]
          }
        }
      }
    }  
  ]  
);
\end{lstlisting}

\begin{lstlisting}
{ "_id" : 5555, "facturacion_total" : 1580 }
{ "_id" : 2222, "facturacion_total" : 6327 }
{ "_id" : 1111, "facturacion_total" : 830 }
\end{lstlisting}

\section{Objetivo nº 3. MapReduce}

Vamos a utilizar la base de datos libre GeoWorldMap de GeoBytes. Es una base de datos de países, con sus estados/regiones y ciudades importantes. Sobre esta base de datos vamos a obtener el par de ciudades que se encuentran más cercanas de cada país, excluyendo a los EEUU.
\\ \\
Vamos a importar en nuestra BD de MongoDB un archivo de 37245 ciudades del mundo que está en formato csv (\texttt{/tmp/mongo/Cities.csv})

\begin{lstlisting}
mongoimport -u <user> -p <clave> --db <bd> --collection cities --type csv --headerline --file /home/Cities.csv
\end{lstlisting}

\subsection{Ejercicio 1}

Encontrar las ciudades más cercanas sobre la colección recién creada mediante un enfoque MapReduce conforme a los pasos que se ilustran en el tutorial práctico.

\begin{lstlisting}
var mapCode = function() {
  emit(
    this.CountryID,
    { 
      "data":
      [
        {
          "city": this.City,
          "lat":  this.Latitude,
          "lon":  this.Longitude
        }
      ]
    }
  );
}

var reduceCode = function(key, values) {
  var reduced = { 
    "data": [] 
  };
  for (var i in values) {
    var inter = values[i];
    for (var j in inter.data) {
      reduced.data.push(inter.data[j]);
    }
  }
  return reduced;
}

var finalize =  function (key, reduced) {
  if (reduced.data.length == 1) {
    return { 
      "message" : "Este pais solo contiene una ciudad" 
    };
  }
  var min_dist = 999999999999;
  var city1 = { 
    "city": ""
  };
  var city2 = { 
    "city": "" 
  };
  var c1;
  var c2;
  var d;
  for (var i in reduced.data) {
    for (var j in reduced.data) {
      if (i >= j) {
        continue;
      }
      c1 = reduced.data[i];
      c2 = reduced.data[j];
      d = (c1.lat - c2.lat) * (c1.lat - c2.lat) + (c1.lon - c2.lon) * (c1.lon - c2.lon);
      if (d < min_dist && d > 0) {
        min_dist = d;
        city1 = c1;
        city2 = c2;
      }
    }
  }
  return {
    "city1": city1.city, 
    "city2": city2.city, 
    "dist": Math.sqrt(min_dist) 
  };
}

db.runCommand({
  mapReduce: "cities",
  map: mapCode,
  reduce: reduceCode,
  finalize: finalize,
  query: { CountryID: { $ne: 254 } },
  out: { merge: "ciudades_proximas" }
});
\end{lstlisting}

\begin{lstlisting}
{
  "result" : "ciudades_proximas",
  "timeMillis" : 1009,
  "counts" : {
    "input" : 7042,
    "emit" : 7042,
    "reduce" : 260,
    "output" : 215
  },
  "ok" : 1
}
\end{lstlisting}

\subsection{Ejercicio 2}

¿Cómo podríamos obtener las ciudades más distintas en cada país?

\begin{lstlisting}
var mapCode = function() {
  emit(
    this.CountryID,
    { 
      "data":
      [
        {
          "city": this.City,
          "lat":  this.Latitude,
          "lon":  this.Longitude
        }
      ]
    }
  );
}

var reduceCode = function(key, values) {
  var reduced = { 
    "data": [] 
  };
  for (var i in values) {
    var inter = values[i];
    for (var j in inter.data) {
      reduced.data.push(inter.data[j]);
    }
  }
  return reduced;
}

var finalize =  function (key, reduced) {
  if (reduced.data.length == 1) {
    return { 
      "message" : "Este pais solo contiene una ciudad" 
    };
  }
  var max_dist = 0;
  var city1 = { 
    "city": ""
  };
  var city2 = { 
    "city": "" 
  };
  var c1;
  var c2;
  var d;
  for (var i in reduced.data) {
    for (var j in reduced.data) {
      if (i >= j) {
        continue;
      }
      c1 = reduced.data[i];
      c2 = reduced.data[j];
      d = (c1.lat - c2.lat) * (c1.lat - c2.lat) + (c1.lon - c2.lon) * (c1.lon - c2.lon);
      if (d > max_dist && d > 0) {
        max_dist = d;
        city1 = c1;
        city2 = c2;
      }
    }
  }
  return {
    "city1": city1.city, 
    "city2": city2.city, 
    "dist": Math.sqrt(max_dist) 
  };
}

db.runCommand({
  mapReduce: "cities",
  map: mapCode,
  reduce: reduceCode,
  finalize: finalize,
  query: { CountryID: { $ne: 254 } },
  out: { merge: "ciudades_lejanas" }
});
\end{lstlisting}

\begin{lstlisting}
{
  "result" : "ciudades_lejanas",
  "timeMillis" : 1046,
  "counts" : {
    "input" : 7042,
    "emit" : 7042,
    "reduce" : 260,
    "output" : 214
  },
  "ok" : 1
}
\end{lstlisting}

\subsection{Ejercicio 3}

¿Qué ocurre si en un país hay dos parejas de ciudades que están a la misma distancia mínima? ¿Cómo harías para que aparecieran todas?

\begin{lstlisting}
var mapCode = function() {
  emit(
    this.CountryID,
    { 
      "data":
      [
        {
          "city": this.City,
          "lat":  this.Latitude,
          "lon":  this.Longitude
        }
      ]
    }
  );
}

var reduceCode = function(key, values) {
  var reduced = { 
    "data": [] 
  };
  for (var i in values) {
    var inter = values[i];
    for (var j in inter.data) {
      reduced.data.push(inter.data[j]);
    }
  }
  return reduced;
}

var finalize =  function (key, reduced) {
  if (reduced.data.length == 1) {
    return { 
      "message" : "Este pais solo contiene una ciudad" 
    };
  }
  var min_dist = 999999999999;
  var cities = [];
  var c1;
  var c2;
  var d;
  for (var i in reduced.data) {
    for (var j in reduced.data) {
      if (i >= j) {
        continue;
      }
      c1 = reduced.data[i];
      c2 = reduced.data[j];
      d = (c1.lat - c2.lat) * (c1.lat - c2.lat) + (c1.lon - c2.lon) * (c1.lon - c2.lon);
      if (d < min_dist && d > 0) {
        min_dist = d;
        cities = [];
        cities.push([c1.city, c2.city]);
      } else if (d == min_dist) {
        cities.push([c1.city, c2.city]);
      }
    }
  }
  return {
    "cities": cities,
    "dist": Math.sqrt(min_dist) 
  };
}

db.runCommand({
  mapReduce: "cities",
  map: mapCode,
  reduce: reduceCode,
  finalize: finalize,
  query: { CountryID: { $ne: 254 } },
  out: { merge: "ciudades_proximas" }
});
\end{lstlisting}

\begin{lstlisting}
{
  "result" : "ciudades_proximas",
  "timeMillis" : 1120,
  "counts" : {
    "input" : 7042,
    "emit" : 7042,
    "reduce" : 260,
    "output" : 215
  },
  "ok" : 1
}
\end{lstlisting}

\subsection{Ejercicio 4}

¿Cómo podríamos obtener adicionalmente la cantidad de parejas de ciudades evaluadas para cada país consultado?

\begin{lstlisting}
var mapCode = function() {
  emit(
    this.CountryID,
    { 
      "data":
      [
        {
          "city": this.City,
          "lat":  this.Latitude,
          "lon":  this.Longitude
        }
      ]
    }
  );
}

var reduceCode = function(key, values) {
  var reduced = { 
    "data": [] 
  };
  for (var i in values) {
    var inter = values[i];
    for (var j in inter.data) {
      reduced.data.push(inter.data[j]);
    }
  }
  return reduced;
}

var finalize =  function (key, reduced) {
  if (reduced.data.length == 1) {
    return { 
      "message" : "Este pais solo contiene una ciudad" 
    };
  }
  var min_dist = 999999999999;
  var city1 = { 
    "city": ""
  };
  var city2 = { 
    "city": "" 
  };
  var c1;
  var c2;
  var d;
  var eval = 0;
  for (var i in reduced.data) {
    for (var j in reduced.data) {
      if (i >= j) {
        continue;
      }
      c1 = reduced.data[i];
      c2 = reduced.data[j];
      d = (c1.lat - c2.lat) * (c1.lat - c2.lat) + (c1.lon - c2.lon) * (c1.lon - c2.lon);
      if (d > 0) {
        eval++;
        if (d < min_dist) {
          min_dist = d;
          city1 = c1;
          city2 = c2;
        }
      }
    }
  }
  return {
    "city1": city1.city, 
    "city2": city2.city, 
    "cities": eval,
    "dist": Math.sqrt(min_dist) 
  };
}

db.runCommand({
  mapReduce: "cities",
  map: mapCode,
  reduce: reduceCode,
  finalize: finalize,
  query: { CountryID: { $ne: 254 } },
  out: { merge: "ciudades_proximas" }
});
\end{lstlisting}

\begin{lstlisting}
{
  "result" : "ciudades_proximas",
  "timeMillis" : 1038,
  "counts" : {
    "input" : 7042,
    "emit" : 7042,
    "reduce" : 260,
    "output" : 215
  },
  "ok" : 1
}
\end{lstlisting}

\subsection{Ejercicio 5}

¿Cómo podríamos la distancia media entre las ciudades de cada país?

\begin{lstlisting}
var mapCode = function() {
  emit(
    this.CountryID,
    { 
      "data":
      [
        {
          "city": this.City,
          "lat":  this.Latitude,
          "lon":  this.Longitude
        }
      ]
    }
  );
}

var reduceCode = function(key, values) {
  var reduced = { 
    "data": [] 
  };
  for (var i in values) {
    var inter = values[i];
    for (var j in inter.data) {
      reduced.data.push(inter.data[j]);
    }
  }
  return reduced;
}

var finalize =  function (key, reduced) {
  if (reduced.data.length == 1) {
    return { 
      "message" : "Este pais solo contiene una ciudad" 
    };
  }
  var min_dist = 999999999999;
  var city1 = { 
    "city": ""
  };
  var city2 = { 
    "city": "" 
  };
  var c1;
  var c2;
  var d;
  var eval = 0;
  var tot = 0;
  for (var i in reduced.data) {
    for (var j in reduced.data) {
      if (i >= j) {
        continue;
      }
      c1 = reduced.data[i];
      c2 = reduced.data[j];
      d = Math.sqrt((c1.lat - c2.lat) * (c1.lat - c2.lat) + (c1.lon - c2.lon) * (c1.lon - c2.lon));
      if (d > 0) {
        eval++;
        tot += d;
        if (d < min_dist) {
          min_dist = d;
          city1 = c1;
          city2 = c2;
        }
      }
    }
  }
  return {
    "city1": city1.city, 
    "city2": city2.city, 
    "dist_avg": tot / eval,
    "dist": min_dist
  };
}

db.runCommand({
  mapReduce: "cities",
  map: mapCode,
  reduce: reduceCode,
  finalize: finalize,
  query: { CountryID: { $ne: 254 } },
  out: { merge: "ciudades_proximas" }
});
\end{lstlisting}

\begin{lstlisting}
{
  "result" : "ciudades_proximas",
  "timeMillis" : 1073,
  "counts" : {
    "input" : 7042,
    "emit" : 7042,
    "reduce" : 260,
    "output" : 215
  },
  "ok" : 1
}
\end{lstlisting}

\subsection{Ejercicio 6}

¿Mejoraría el rendimiento si creamos un índice? ¿Sobre qué campo? Comprobadlo

\begin{lstlisting}
db.cities.ensureIndex({CountryID: 1});

var mapCode = function() {
  emit(
    this.CountryID_1,
    { 
      "data":
      [
        {
          "city": this.City,
          "lat":  this.Latitude,
          "lon":  this.Longitude
        }
      ]
    }
  );
}

var reduceCode = function(key, values) {
  var reduced = { 
    "data": [] 
  };
  for (var i in values) {
    var inter = values[i];
    for (var j in inter.data) {
      reduced.data.push(inter.data[j]);
    }
  }
  return reduced;
}

var finalize =  function (key, reduced) {
  if (reduced.data.length == 1) {
    return { 
      "message" : "Este pais solo contiene una ciudad" 
    };
  }
  var min_dist = 999999999999;
  var city1 = { 
    "city": ""
  };
  var city2 = { 
    "city": "" 
  };
  var c1;
  var c2;
  var d;
  for (var i in reduced.data) {
    for (var j in reduced.data) {
      if (i >= j) {
        continue;
      }
      c1 = reduced.data[i];
      c2 = reduced.data[j];
      d = (c1.lat - c2.lat) * (c1.lat - c2.lat) + (c1.lon - c2.lon) * (c1.lon - c2.lon);
      if (d < min_dist && d > 0) {
        min_dist = d;
        city1 = c1;
        city2 = c2;
      }
    }
  }
  return {
    "city1": city1.city, 
    "city2": city2.city, 
    "dist": Math.sqrt(min_dist) 
  };
}

db.runCommand({
  mapReduce: "cities",
  map: mapCode,
  reduce: reduceCode,
  finalize: finalize,
  query: { CountryID: { $ne: 254 } },
  out: { merge: "ciudades_proximas" }
});
\end{lstlisting}

\begin{lstlisting}
{
  "result" : "ciudades_proximas",
  "timeMillis" : 12389,
  "counts" : {
    "input" : 7042,
    "emit" : 7042,
    "reduce" : 72,
    "output" : 216
  },
  "ok" : 1
}
\end{lstlisting}

Como podemos ver, la introducción de un índice no solo no mejora, sino que empeora los tiempos de ejecución en torno a un 1000\%.

%----------------------------------------------------------------------------------------
%	REFERENCIAS
%----------------------------------------------------------------------------------------

%\newpage

%\bibliography{referencias} %archivo referencias.bib que contiene las entradas 
%\bibliographystyle{plain} % hay varias formas de citar

\end{document}
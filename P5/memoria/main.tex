\input{estilo.tex}
\usepackage{textcomp}
\usepackage{hyperref}

%----------------------------------------------------------------------------------------
%	DATOS
%----------------------------------------------------------------------------------------

\newcommand{\myName}{Francisco Javier Bolívar Lupiáñez}
\newcommand{\myDegree}{Máster en Ingeniería Informática}
\newcommand{\myFaculty}{E. T. S. de Ingenierías Informática y de Telecomunicación}
\newcommand{\myDepartment}{Ciencias de la Computación e Inteligencia Artificial}
\newcommand{\myUniversity}{\protect{Universidad de Granada}}
\newcommand{\myLocation}{Granada}
\newcommand{\myTime}{\today}
\newcommand{\myTitle}{Práctica 5}
\newcommand{\mySubtitle}{Ciencia de Datos con Hadoop}
\newcommand{\mySubject}{Cloud Computing: Servicios y Aplicaciones}
\newcommand{\myYear}{2016-2017}

%----------------------------------------------------------------------------------------
%	PORTADA
%----------------------------------------------------------------------------------------


\title{	
	\normalfont \normalsize 
	\textsc{\textbf{\mySubject \space (\myYear)} \\ \myDepartment} \\[20pt] % Your university, school and/or department name(s)
	\textsc{\myDegree \\[10pt] \myFaculty \\ \myUniversity} \\[25pt]
	\horrule{0.5pt} \\[0.4cm] % Thin top horizontal rule
	\huge \myTitle: \mySubtitle \\ % The assignment title
	\horrule{2pt} \\[0.5cm] % Thick bottom horizontal rule
	\normalfont \normalsize
}

\author{\myName} % Nombre y apellidos

\date{\myTime} % Incluye la fecha actual
%----------------------------------------------------------------------------------------
%	INDICE
%----------------------------------------------------------------------------------------

\begin{document}
	
\lstset {
	basicstyle=\scriptsize,
	frame=single,
	backgroundcolor=\color{grey},
	breaklines=true
}
	
\setcounter{page}{0}

\maketitle % Muestra el Título
\thispagestyle{empty}

\newpage %inserta un salto de página

\tableofcontents % para generar el índice de contenidos

%\listoffigures

\newpage

%----------------------------------------------------------------------------------------
%	DOCUMENTO
%----------------------------------------------------------------------------------------

\section{Introducción}

El objetivo de esta práctica es conocer las alternativas para realizar experimentos de Ciencia de Datos. Para ello, haremos uso del entorno que se ha convertido en un estándar de facto como es Hadoop, utilizando HDFS como sistema de archivos distribuido y Hadoop-MapReduce como mecanismo de ejecución. Por último aplicaremos la biblioteca Mahout para lanzar algoritmos de clasificación sobre conjuntos tipo Big Data.
\\ \\
Para constatar el manejo de la herramienta anterior, el alumno deberá realizar las tareas que se describen a continuación y entregar documentación escribiendo las tareas realizadas.

\section{Tareas}

Las tareas a realizar serán las siguientes:

\begin{itemize}
	\item Ejecutar el algoritmo "Random Forest" sobre el conjunto de datos BNG\_heart y comprueba el rendimiento alcanzado de acuerdo a los siguientes casos:
	\begin{itemize}
		\item Número de Maps: 64, 128, 256
		\item Número de árboles: 10, 100, 1000
	\end{itemize}
	\item Del punto anterior, obtener una tabla que indique los siguientes datos:
	\begin{itemize}
		\item Características del modelo: Número de nodos (total y promedio), profundidad máxima del árbol.
		\item Tiempo de ejecución para entrenamiento.
		\item Medidas de calidad Accuracy estándar y media geométrica tanto para la partición de entrenamiento como para test.
	\end{itemize}
\end{itemize}

\newpage
\subsection{Características del modelo}

\subsubsection{Número de nodos total}

\begin{table}[H]
	\centering
	\caption{Número de nodos total para 64, 128 y 256 maps y 10, 100 y 1000 árboles}
	\label{tab:nodos-total}
	\begin{tabular}{l|lll|}
		\cline{2-4}
		& 10    & 100    & 1000    \\ \hline
		\multicolumn{1}{|l|}{64}  & 13014 & 128674 & 1285305 \\
		\multicolumn{1}{|l|}{128} & 6974  & 70146  & 681270  \\
		\multicolumn{1}{|l|}{256} & 3671  & 36593  & 362281  \\ \hline
	\end{tabular}
\end{table}

\begin{figure}[H]
	\centering
	\includegraphics[width=12cm]{img/nodos-total}
	\caption{Número de nodos total para 64, 128 y 256 maps y 10, 100 y 1000 árboles}
	\label{fig:nodos-total}
\end{figure}

Como vemos el número de nodos total depende, sobre todo, del número de árboles (a mayor número de árboles mayor número de nodos) aunque también depende del número de maps (a mayor número de maps, menor número de nodos).

\newpage
\subsubsection{Número de nodos promedio}

\begin{table}[H]
	\centering
	\caption{Número de nodos promedio para 64, 128 y 256 maps y 10, 100 y 1000 árboles}
	\label{tab:nodos-promedio}
	\begin{tabular}{l|lll|}
		\cline{2-4}
		& 10    & 100    & 1000    \\ \hline
		\multicolumn{1}{|l|}{64}  & 1301 & 1286 & 1285 \\
		\multicolumn{1}{|l|}{128} & 694  & 701  & 681  \\
		\multicolumn{1}{|l|}{256} & 367  & 365  & 362  \\ \hline
	\end{tabular}
\end{table}

\begin{figure}[H]
	\centering
	\includegraphics[width=12cm]{img/nodos-promedio}
	\caption{Número de nodos promedio para 64, 128 y 256 maps y 10, 100 y 1000 árboles}
	\label{fig:nodos-promedio}
\end{figure}

Como vemos el número de nodos promedio del número de maps (a mayor número de maps, menor número de nodos). En este caso, el número de árboles no es un factor tan determinante.

\newpage
\subsubsection{Profundidad máxima del árbol}

\begin{table}[H]
	\centering
	\caption{Profundidad máxima para 64, 128 y 256 maps y 10, 100 y 1000 árboles}
	\label{tab:profundidad-maxima}
	\begin{tabular}{l|lll|}
		\cline{2-4}
		& 10    & 100    & 1000    \\ \hline
		\multicolumn{1}{|l|}{64}  & 19 & 18 & 18 \\
		\multicolumn{1}{|l|}{128} & 15 & 16 & 15 \\
		\multicolumn{1}{|l|}{256} & 13 & 14 & 13 \\ \hline
	\end{tabular}
\end{table}

\begin{figure}[H]
	\centering
	\includegraphics[width=12cm]{img/profundidad-maxima}
	\caption{Profundidad máxima para 64, 128 y 256 maps y 10, 100 y 1000 árboles}
	\label{fig:profundidad-maxima}
\end{figure}

Como vemos, al igual que con el número de nodos promedio, la profundidad máxima del árbol depende del número de maps (a mayor número de maps, menor número de nodos).

\newpage
\subsection{Tiempo de ejecución para entrenamiento}

\begin{table}[H]
	\centering
	\caption{Tiempo de ejecución para entrenamiento para 64, 128 y 256 maps y 10, 100 y 1000 árboles}
	\label{tab:tiempo-entrenamiento}
	\begin{tabular}{l|lll|}
		\cline{2-4}
		& 10    & 100    & 1000    \\ \hline
		\multicolumn{1}{|l|}{64}  & 18.915 & 33.955 & 44.287 \\
		\multicolumn{1}{|l|}{128} & 21.28  & 31.201 & 34.635 \\
		\multicolumn{1}{|l|}{256} & 30.97  & 31.326 & 33.59  \\ \hline
	\end{tabular}
\end{table}

\begin{figure}[H]
	\centering
	\includegraphics[width=12cm]{img/tiempo-entrenamiento}
	\caption{Tiempo de ejecución para entrenamiento para 64, 128 y 256 maps y 10, 100 y 1000 árboles}
	\label{fig:tiempo-entrenamiento}
\end{figure}

En el caso del tiempo el comportamiento es curioso porque actúa de forma distinta dependiendo del número de árboles:
\begin{itemize}
	\item Si hay pocos árboles (10), a mayor sea el número de maps, mayor será el tiempo.
	\item Si hay muchos árboles (100 y 1000), a mayor sea el número de maps, menor será el tiempo. Este efecto se nota más todavía a mayor sea el número de árboles.
\end{itemize}

\newpage
\subsection{Medidas de calidad}

\subsubsection{Accuracy estándar (test)}

\begin{table}[H]
	\centering
	\caption{Accuracy estándar sobre la partición de test para entrenamiento para 64, 128 y 256 maps y 10, 100 y 1000 árboles}
	\label{tab:accuracy-test}
	\begin{tabular}{l|lll|}
		\cline{2-4}
		& 10    & 100    & 1000    \\ \hline
		\multicolumn{1}{|l|}{64}  & 87.768  & 89.418  & 89.674  \\
		\multicolumn{1}{|l|}{128} & 87.3175 & 88.1775 & 89.43   \\
		\multicolumn{1}{|l|}{256} & 86.848  & 87.787  & 89.2015 \\ \hline
	\end{tabular}
\end{table}

\begin{figure}[H]
	\centering
	\includegraphics[width=12cm]{img/accuracy-test}
	\caption{Accuracy estándar sobre la partición de test para 64, 128 y 256 maps y 10, 100 y 1000 árboles}
	\label{fig:accuracy-test}
\end{figure}

El número de árboles influye mucho en el resultado. A mayor sea mejor resultado se obtendrá. También influye positivamente un mayor número de maps.

\newpage
\subsubsection{Accuracy estándar (training)}

\begin{table}[H]
	\centering
	\caption{Accuracy estándar sobre la partición de training para entrenamiento para 64, 128 y 256 maps y 10, 100 y 1000 árboles}
	\label{tab:accuracy-training}
	\begin{tabular}{l|lll|}
		\cline{2-4}
		& 10    & 100    & 1000    \\ \hline
		\multicolumn{1}{|l|}{64}  & 87.7676 & 89.5021 & 89.7058 \\
		\multicolumn{1}{|l|}{128} & 87.3186 & 88.281  & 89.4641 \\
		\multicolumn{1}{|l|}{256} & 86.8689 & 87.6859 & 89.2154 \\ \hline
	\end{tabular}
\end{table}

\begin{figure}[H]
	\centering
	\includegraphics[width=12cm]{img/accuracy-training}
	\caption{Accuracy estándar sobre la partición de training para 64, 128 y 256 maps y 10, 100 y 1000 árboles}
	\label{fig:accuracy-training}
\end{figure}

Los resultados obtenidos sobre el conjunto de entrenamiento son algo mejores a los obtenidos con el conjunto de test, no obstante son muy parecidos.

\newpage
\subsubsection{Media geométrica (test)}

\begin{table}[H]
	\centering
	\caption{Media geométrica sobre la partición de test para entrenamiento para 64, 128 y 256 maps y 10, 100 y 1000 árboles}
	\label{tab:media-geometrica-test}
	\begin{tabular}{l|lll|}
		\cline{2-4}
		& 10    & 100    & 1000    \\ \hline
		\multicolumn{1}{|l|}{64}  & 0.87518301 & 0.89188743 & 0.8943475  \\
		\multicolumn{1}{|l|}{128} & 0.87026243 & 0.87948937 & 0.89185739 \\
		\multicolumn{1}{|l|}{256} & 0.86691769 & 0.87623183 & 0.88985665 \\ \hline
	\end{tabular}
\end{table}

\begin{figure}[H]
	\centering
	\includegraphics[width=12cm]{img/media-geometrica-test}
	\caption{Media geométrica sobre la partición de test para 64, 128 y 256 maps y 10, 100 y 1000 árboles}
	\label{fig:media-geometrica-test}
\end{figure}

Con la media geométrica el comportamiento es muy similar al obtenido con el accuracy.

\newpage
\subsubsection{Media geométrica (training)}

\begin{table}[H]
	\centering
	\caption{Media geométrica sobre la partición de training para entrenamiento para 64, 128 y 256 maps y 10, 100 y 1000 árboles}
	\label{tab:media-geometrica-training}
	\begin{tabular}{l|lll|}
		\cline{2-4}
		& 10    & 100    & 1000    \\ \hline
		\multicolumn{1}{|l|}{64}  & 0.87556729 & 0.89285623 & 0.89478267 \\
		\multicolumn{1}{|l|}{128} & 0.87094252 & 0.88039306 & 0.8923021  \\
		\multicolumn{1}{|l|}{256} & 0.86500017 & 0.87518159 & 0.89018632 \\ \hline
	\end{tabular}
\end{table}

\begin{figure}[H]
	\centering
	\includegraphics[width=12cm]{img/media-geometrica-training}
	\caption{Media geométrica sobre la partición de training para 64, 128 y 256 maps y 10, 100 y 1000 árboles}
	\label{fig:media-geometrica-training}
\end{figure}

Al igual que con el accuracy, la diferencia entre la media geométrica sobre el conjunto de training no difiere mucho a la obtenida sobre el conjunto de test.

%----------------------------------------------------------------------------------------
%	REFERENCIAS
%----------------------------------------------------------------------------------------

%\newpage

%\bibliography{referencias} %archivo referencias.bib que contiene las entradas 
%\bibliographystyle{plain} % hay varias formas de citar

\end{document}
\input{estilo.tex}
\usepackage{textcomp}
\usepackage{hyperref}

%----------------------------------------------------------------------------------------
%	DATOS
%----------------------------------------------------------------------------------------

\newcommand{\myName}{Francisco Javier Bolívar Lupiáñez}
\newcommand{\myDegree}{Máster en Ingeniería Informática}
\newcommand{\myFaculty}{E. T. S. de Ingenierías Informática y de Telecomunicación}
\newcommand{\myDepartment}{Ciencias de la Computación e Inteligencia Artificial}
\newcommand{\myUniversity}{\protect{Universidad de Granada}}
\newcommand{\myLocation}{Granada}
\newcommand{\myTime}{\today}
\newcommand{\myTitle}{Práctica 4}
\newcommand{\mySubtitle}{Computación Distribuida y Escalable con Hadoop}
\newcommand{\mySubject}{Cloud Computing: Servicios y Aplicaciones}
\newcommand{\myYear}{2016-2017}

%----------------------------------------------------------------------------------------
%	PORTADA
%----------------------------------------------------------------------------------------


\title{	
	\normalfont \normalsize 
	\textsc{\textbf{\mySubject \space (\myYear)} \\ \myDepartment} \\[20pt] % Your university, school and/or department name(s)
	\textsc{\myDegree \\[10pt] \myFaculty \\ \myUniversity} \\[25pt]
	\horrule{0.5pt} \\[0.4cm] % Thin top horizontal rule
	\huge \myTitle: \mySubtitle \\ % The assignment title
	\horrule{2pt} \\[0.5cm] % Thick bottom horizontal rule
	\normalfont \normalsize
}

\author{\myName} % Nombre y apellidos

\date{\myTime} % Incluye la fecha actual
%----------------------------------------------------------------------------------------
%	INDICE
%----------------------------------------------------------------------------------------

\begin{document}
	
\lstset {
	basicstyle=\scriptsize,
	frame=single,
	backgroundcolor=\color{grey},
	breaklines=true
}
	
\setcounter{page}{0}

\maketitle % Muestra el Título
\thispagestyle{empty}

\newpage %inserta un salto de página

\tableofcontents % para generar el índice de contenidos

%\listoffigures

\newpage

%----------------------------------------------------------------------------------------
%	DOCUMENTO
%----------------------------------------------------------------------------------------

\section{Introducción}

El objetivo de esta práctica es realizar programas escalables para mejorar la eficiencia en entornos Big Data. Para ello, haremos uso del entorno que se ha convertido en un estándar de facto como es Hadoop, utilizando HDFS como sistema de archivos distribuido y Hadoop-MapReduce como mecanismo de ejecución.
\\ \\
Para constatar el manejo de la herramienta anterior, el alumno deberá las realizar tareas que se describen a continuación y entregar documentación describiendo las tareas realizadas.

\section{Tareas}

Utilizando como base el conjunto de datos ECBDL14 situado en la carpeta \texttt{/tmp/BDCC/} \texttt{datasets/EDBDL14/ECBDL14\_10tst.data} obtener los siguientes datos estadísticos descriptivos.

\subsection{Tarea 1}

Calcular el valor mínimo de la variable (columna) 5:
\\ \\
El código está en la carpeta \texttt{Min}.

\begin{lstlisting}
Min -11.0
\end{lstlisting}

\subsection{Tarea 2}

Calcular el valor máximo de la variable (columna) 5:
\\ \\
El código está en la carpeta \texttt{Max}.

\begin{lstlisting}
Max 9.0
\end{lstlisting}

\subsection{Tarea 3}

Calcular al mismo tiempo los valores máximo y mínimo de la variable 5:
\\ \\
El código está en la carpeta \texttt{MaxMin}.

\begin{lstlisting}
Max 9.0
Min -11.0
\end{lstlisting}

\subsection{Tarea 4}

Calcula los valores máximo y mínimo de todas las variables (salvo la última, que es la etiqueta de clase):
\\ \\
El código está en la carpeta \texttt{MaxMinAll}.

\begin{lstlisting}
Max (1) 0.154
Min (1) 0.0
Max (2) 10.0
Min (2) -12.0
Max (3) 8.0
Min (3) -11.0
Max (4) 9.0
Min (4) -12.0
Max (5) 9.0
Min (5) -11.0
Max (6) 9.0
Min (6) -13.0
Max (7) 9.0
Min (7) -12.0
Max (8) 7.0
Min (8) -12.0
Max (9) 10.0
Min (9) -13.0
Max (0) 0.768
Min (0) 0.094
\end{lstlisting}

\subsection{Tarea 5}

Realizar la media de la variable 5:
\\ \\
El código está en la carpeta \texttt{Avg}.

\begin{lstlisting}
Avg -1.282261707288373
\end{lstlisting}

\subsection{Tarea 6}

Obtener la media de todas las variables (salvo la clase):
\\ \\
El código está en la carpeta \texttt{AvgAll}.

\begin{lstlisting}
1   0.05212776590940497
2   -2.188240380935686
3   -1.408876789776933
4   -1.7528724942777865
5   -1.282261707288373
6   -2.293434905140485
7   -1.5875789403216172
8   -1.7390052924221087
9   -1.6989002790625127
0   0.25496195991730947
\end{lstlisting}

\subsection{Tarea 7}

Comprobar si el conjunto de datos ECBDL es balanceado o no balanceado, es decir, que el ratio entre las clases sea menor o mayor que 1.5 respectivamente:
\\ \\
El código está en la carpeta \texttt{Bal}.

\begin{lstlisting}
Bal 58.582560602010815
\end{lstlisting}

Al ser el ratio mayor a 1.5 podemos determinar que el conjunto de datos no es balanceado.

\subsection{Tarea 8}

Cálculo del coeficiente de correlación entre todas las parejas de variables:

\begin{lstlisting}

\end{lstlisting}

%----------------------------------------------------------------------------------------
%	REFERENCIAS
%----------------------------------------------------------------------------------------

%\newpage

%\bibliography{referencias} %archivo referencias.bib que contiene las entradas 
%\bibliographystyle{plain} % hay varias formas de citar

\end{document}